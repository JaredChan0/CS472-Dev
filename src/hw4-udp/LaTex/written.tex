\documentclass[12pt, a4paper]{article}
\usepackage[T1]{fontenc}
\usepackage{tgbonum}

\title{Programming Assignment 4}
\author{Jared Chandavong}

\begin{document}

\maketitle

\newpage
{\fontfamily{cmss}\selectfont
\section{Function Descriptions}

\subsection{static dp\_connp dpinit()}
This function initializes a Drexel Protocol (DP) connection.

\subsection{void dpclose(dp\_connp dpsession)}
This function takes in a Drexel Protocol (DP) session and closes it by freeing its allocated memory.

\subsection{int dpmaxdgram()}
This function returns the Drexel Protocol's (DP) maximum buffer size. 

\subsection{dp\_connp dpServerInit(int port)}
This function initializes a Drexel Protocol (DP) server over UDP by behaving as follows: \newline
\indent Calls dpinit() to allocate a new dp connection. If initialization fails, it \newline 
\indent\indent prints an error and returns NULL. \newline
\indent Attempts to create a UDP socket. If socket creation fails, it prints an \newline
\indent\indent error and returns NULL. \newline
\indent Fills up the sockaddr\_in structure with appropriate information. \newline
\indent Enables the server to reuse the port. If setting the option fails, it prints \newline
\indent\indent an error, closes the socket, and returns NULL. \newline
\indent Bind the Socket. If binding fails, it prints an error, closes the socket, and \newline
\indent\indent returns NULL. \newline
\indent Sets the dp initialization to true and returns the connection.

\subsection{dp\_connp dpClientInit(char *addr, int port)}
This function initializes a Drexel Protocol (DP) client over UDP by behaving as follows: \newline
\indent Calls dpinit() to allocate a new dp connection. If initialization fails, it \newline 
\indent\indent prints an error and returns NULL. \newline
\indent Attempts to create a UDP socket. If socket creation fails, it prints an \newline
\indent\indent error and returns NULL. \newline
\indent Fills up the sockaddr\_in structure with appropriate information. \newline
\indent Copies the memory of the outbound address into the inbound address. \newline
\indent Returns the connection.

\subsection{int dprecv(dp\_connp dp, void *buff, int buff\_sz)}
This function recieves PDU of an inbound Drexel Protocol (DP) connection by behaving as follows: \newline
\indent Calls dprecvdgram(...) to store the datagram into a buffer and returns a \newline
\indent\indent received length. If the length is -16, then it returns a closed connection \newline
\indent\indent signal. \newline
\indent Convert the datagram into a DP PDU. \newline
\indent If the recieved length is greater than the size of the PDU header, then \newline
\indent\indent copy the memory of the payload into a buffer. \newline
\indent Return the size of the datagram stored in the inbound PDU.

\subsection{static int dprecvdgram(dp\_connp dp, void *buff, int buff\_sz)}
This function recieves a datagram of an inbound Drexel Protocol (DP) connection and validates its by behvaing as follows: \newline
\indent If the buffer size of the datagram is greater than the maximum DP \newline
\indent\indent datagram, then return an error that the DP buffer is oversized. \newline
\indent Calls dprecvraw(...) to get the number of bytes in the inbound datagram. \newline
\indent If the inbound bytes is less than the size of the PDU header, then an \newline
\indent\indent error code is made that the datagram is bad. \newline
\indent Copies the memory of the inbound datagram into a PDU struct. If the \newline 
\indent\indent PDU's datagram size is greater than the buffer size, then an error \newline
\indent\indent code is made that the buffer is undersized. \newline
\indent Updates the sequence number and prepares the acknowledgement packet. \newline
\indent Creates and fills out an outbound PDU to be sent out. \newline
\indent If there is an error code, send a error message. \newline
\indent Handles different message types based off the inbound PDU. If an \newline
\indent\indent error occurs in any case, then return an error messeage. \newline
\indent Returns the number of bytes from the inbound datagram.

\subsection{static int dprecvraw(dp\_connp dp, void *buff, int buff\_sz)}
This function recieves raw UDP datagrams and validates them by behaving as follows: \newline
\indent If the inbound socket address is not initialized, then return an error. \newline
\indent Calls recvfrom(...) to store the datagram into a buffer. \newline
\indent If the received bytes is less than 0, then return an error. \newline
\indent Set the outbound socket address is set to initialized. \newline
\indent Prints the inbound PDU datagram and returns the number of bytes.

\subsection{int dpsend(dp\_connp dp, void *sbuff, int sbuff\_sz)}
This function validates the buffer being sent by behaving as follows: \newline
\indent If the size of the send buffer is larger than the max DP datagram size, \newline
\indent\indent then return an error code that the DP buffer is undersized. \newline
\indent Calls dpsenddgram(...) to get the send size and returns it.

\subsection{static int dpsenddgram(dp\_connp dp, void *sbuff, int sbuff\_sz)}
This function sends a Drexel Protocol (DP) UDP datagram to a receiver by behaving as follows: \newline
\indent If the outbound address is not initialized, then return an error. \newline
\indent If the send buffer size is greater than the max buffer size, then return an \newline
\indent\indent error. \newline
\indent Builds the PDU and outbound buffer. \newline
\indent Copies the memory of the send buffer into the DP buffer. \newline
\indent Calls dpsendraw(...) to get the number of bytes in the outbound. \newline
\indent If the outbound bytes is not equal to the send size, then print a warning. \newline
\indent Updates the sequence number after the send. \newline
\indent Calls dprecvraw(...) to get an ack packet. If the inbound bytes is less \newline
\indent\indent then the size of the PDU header and not an ack packet, then print \newline
\indent\indent a expection error. \newline
\indent Returns the difference between the outbound bytes and PDU header size.

\subsection{static int dpsendraw(dp\_connp dp, void *sbuff, int sbuff\_sz)}
This function sends raw UDP datagrams and sends it to the appropriate receiver by behaving as follows: \newline
\indent If the outbound address is not initialized, then return an error. \newline
\indent Calls sendto(...) to store the bytes into the outbound address. \newline
\indent Calls print\_out\_pdu(...) to print the data in the outbound PDU. \newline
\indent Returns the number of outbound bytes.

\subsection{int dplisten(dp\_connp dp)}
This functions ensures proper Drexel Protocol (DP) connection handshake between client and server by behaving as follows: \newline
\indent If the inbound socket address is not initialized, then return an error. \newline
\indent Calls dprecvraw(...) to receive a connection request. If the received size \newline
\indent\indent is not equal to the size of the PDU header, then return an error. \newline
\indent Calls dpsendraw(...) to send an acknowledgement packet. If the send size \newline
\indent\indent is not equal to size of the PDU header, then return a error. \newline
\indent Set the DP's connection to be true and returns true.

\subsection{int dpconnect(dp\_connp dp)}
This function connects a client to a server by behaving as follows: \newline
\indent If the outbound socket address is not initialized, then return an error. \newline
\indent Calls dpsendraw(...) to send a connection request. If send size is not \newline
\indent\indent equal to the PDU header size, then return an error. \newline
\indent Calls dprecvraw(...) to receive an acknowledgement packet. If receive \newline
\indent\indent size is not equal to the PDU header size, then return an error. \newline
\indent\indent If the recieved messeage type isn't CNTACT, then return an error. \newline
\indent Increments the sequence and set the connection to true. \newline
\indent Returns true.

\subsection{int dpdisconnect(dp\_connp dp)}
This function disconnects a client from a server by behaving as follows: \newline
\indent Builds a close request. \newline
\indent Calls dpsendraw(...) to send a close request. If the send size is not \newline
\indent\indent equal to the PDU header size, then return an error. \newline
\indent Calls dprecvraw(...) to receive an acknowledgement packet. If receive \newline
\indent\indent size is not equal to the PDU header size, then return an error. \newline
\indent\indent If the recieved messeage type isn't CNTACT, then return an error. \newline
\indent Calls dpclose(...) to close the DP connection, and returns closed code.

\subsection{void * dp\_prepare\_send(dp\_pdu *pdu\_ptr, void *buff, int buff\_sz)}
This function prepares a data buffer for sending a Drexel Protocol (DP) message by behaving as follows: \newline
\indent If the buffer size is less than the PDU header size, the return an error. \newline
\indent Sets the bytes in the buffer to 0. \newline
\indent Copies the memory of the PDU to be sent into the buffer. \newline
\indent Returns the sum of the buffer and the PDU header size.

\subsection{void print\_out\_pdu(dp\_pdu *pdu)}
This is a helper for validating the printing of an outbound PDU. It takes a PDU and behaves as follows: \newline
\indent If the debug mode is not equal to 1, then return. \newline
\indent Prints out seperator for PDU details. \newline
\indent Calls print\_pud\_details(...) to print out the PDU's details.

\subsection{void print\_in\_pdu(dp\_pdu *pdu)}
This is a helper for validating the printing of an inbound PDU. It takes a PDU and behaves as follows: \newline
\indent If the debug mode is not equal to 1, then return. \newline
\indent Prints out seperator for PDU details. \newline
\indent Calls print\_pud\_details(...) to print out the PDU's details.

\subsection{static void print\_pdu\_details(dp\_pdu *pdu)}
This is a helper for printing out a PDU's details.

\subsection{static char * pdu\_msg\_to\_string(dp\_pdu *pdu)}
This is a helper that takes a PDU and returns the messeage type it has.

\newpage
\section{3 Sub-layers}

\subsection{Top-Layer}
This layer is comprised of functions such as dpsend() and dprecv() in order to handle
application-level data transfer and receive by preparing and sequencing data.

\subsection{Middle-Layer}
This layer is comprised of functions such as dpsenddgram() and dprecvdgram(). Here, data
is encapsulated into PDUs and deals with the appropriate message types and acknowledgements.

\subsection{Bottom-Layer}
This layer is comprised of functions such as dpsendraw() and dprecvraw(). Raw UDP data is
transfered of the network.

\subsection{Is this good design?}
I believe that this is good design, especially in terms of modulation. Since each layer is contained
within their own functions handling specific responsibilities, it's much easier to error check and
change accordingly. Furthermore, any modification to these functions will only cause isolated effects
that can be used for testing without affecting the other layers.

\section{Sequence Numbers}
In the du-proto, sequence numbers are used to track the order of messages to be processed and to ensure 
that communication between client and server are consistent. As messages are sent to the receiver, ACK
packets are returned with a sequence number for the sender to verify that they have received the correct
message packet. It is also used to handle errors where missing sequence numbers can indicate packet loss. 
\newline \newline
We update the sequence number for ACK responses because since every message has its own sequence number,
we want to avoid a possible issue of the sender ignoring the response for having a duplicate number. 
Furthermore, making sure that each message has a unique sequence number also allows for control messages
to be distinguish from one another so that every packet is processed correctly in order.

\section{ACK vs TCP}
I think that one major example of a limitation of this approach compared to traditional TCP is when
we are transfering large amounts of data. The issue here is that transferring one packet at a time
for an extensively large set of data, as well as waiting for each packet to receive an acknowledgement,
will take a long time to process. In contrast, the traditional TCP approach would be able take on
multiple packets at a time along with acknowledgement for each one, making it generally faster with
reduced overhead.

\section{UDP vs TCP}
When we set up and managed UDP sockets, data is prepared for transfer as discrete packages, or datagrams, 
in an ordered sequence of specified designations before a connection between the sender and the receiver 
is established. In constrast, TCP prepares data as a continuous stream of back and forth responses after
a connection between the sender and the receiver is established. Because of these differences, a key notion
that can be derived is that UDP is faster in transferring data than TCP, but at a cost in that UDP is also 
less reliable in guaranteeing the delivery of packets unlike TCP.

}
\end{document}